\documentclass[12pt]{article}
\usepackage[utf8]{inputenc}
\usepackage{graphicx,stmaryrd}
\usepackage{amsmath,amsfonts,amssymb}
\usepackage{bbold,bbm}
\makeatletter
\def\amsbb{\use@mathgroup \M@U \symAMSb}
\makeatother
\usepackage[cal=boondoxo]{mathalfa}
\newcommand{\typesemi}{\mathbin{\text{\textbb{;}}}}
\newcommand{\typesubset}{\mathbin{\mathbbm{\sqsubseteq}}}
\newcommand{\typeunion}{\mathbin{\mathbb{U}}}
\newcommand{\typeinter}{\mathbin{\text{\rotatebox[origin=c]{180}{$\typeunion$}}}}
\newcommand{\typevee}{\mathbbm{T}}
\newcommand{\typenil}{\mathbbm{F}}
\newcommand{\typecomp}[1]{\neg{#1}}
\newcommand{\typeconv}[1]{{#1}^\smallsmile}
\newcommand{\typeident}{\mathbb{1}}
\newcommand{\typetyped}[2]{{#1}_{\llbracket #2 \rrbracket}}
\newcommand{\conv}[1]{{#1}^{-1}}
\usepackage{algorithmic}
\begin{document}

\title{What's in a name? Disambiguation strategies for relational descriptions}
\author{Sebastiaan Joosten}
\date{Work in progress}
\maketitle

\section{Introduction}
% @Bas, jouw introduction poogt via een omweg bij het op te lossen probleem te komen.
% Een meer directe inleiding zou zijn:

% In a heterogeneous relation algebra~\cite{Hattensperger}, all relations have a type.
% To refer to a relation, one must write $r_{[A\times B]}$ instead of just $r$.
% When the heterogeneous relation algebra is used as a programming language, e.g.~Ampersand~\cite{Ampersand}, identifiers are involved.
% This yields verbose expressions such as:
% \begin{verbatim}
%     account[Person*Bankaccount];creditline[Bankaccount*Amount]
% \end{verbatim}
% A programmer, however, might just want to write \verb-account;creditline-.

% This paper proposes a way to derive the types in expressions of heterogeneous relation algebra,
% to enable more compact references to relations.
% The idea is that type information is only necessary if an expression would otherwise be ambiguous.
% If the programmer already uses the notation $r_{[A\times B]}$, there is no ambiguity.
% But if the programmer refers to a relation just by $r$, and there are multiple relations with that name, ambiguity lures.
% This paper tries to resolve as many ambiguities as possible.

Notations matter: They help communicate ideas by establishing expectations.
When programming, it is recommended to choose function symbols, variable names, etcetera, in a way that it helps the reader of the program.
Notations refer to a certain semantical notion, and choosing a good notation can help establish that.
Regardless of expectations, the thing a notation refers to must always be clear.

Programming languages greatly influence the notations a programmer uses through the rules on what a notation might refer to.
This paper looks at how identifiers that refer to binary relations might be used in a semantic modeling language.
In particular, we look at the use of identifiers in the programming language Ampersand, and ways in which changes to the language might affect the use of identifiers.

Modeling in Ampersand happens via the use of binary relations, which are declared by stating a name, a source concept, and a target concept.
When referring to binary relations, we typically only refer to them by their name.
It is desirable to allow different binary relations to have the same name:
Allowing this gives the programmer greater flexibility in which names to use.

\subsection*{Running example}
An application can be described by binary relations over concepts, a description of what the concepts and relations mean, stating invariants and process rules that determine the behavior of the system, and describing interfaces to interact with the system.
For the purposes of this article, we just declare binary relations and some invariants.

We give the following planning application as an example in which two binary relations share a name:
\begin{verbatim}
before : Task * Task
before : Time * Time
scheduled : Event * Time
\end{verbatim}

An invariant in this system is:

\[
\mathrm{before}_{\mathrm{Task * Task}}\typesemi\mathrm{scheduled}_{\mathrm{Event * Time}} \typesubset \mathrm{scheduled}_\mathrm{Event * Time}\typesemi\mathrm{before}_{\mathrm{Time * Time}}
\]

In heterogeneous relation algebra, where composition is partial, such an invariant would not be well-defined: A Task cannot be compared to an Event, so there is a type error in $\mathrm{before}_{\mathrm{Task * Task}}\typesemi\mathrm{scheduled}_{\mathrm{Event * Time}}$.
However, Ampersand allows sub-typing.
The term $\mathrm{before}_{\mathrm{Task * Task}}\typesemi\mathrm{scheduled}_{\mathrm{Event * Time}}$ will be accepted if there is a least upper bound for the concepts Event and Task.
In this example, we will say that Event is that least upper bound by stating that every Task is an Event.
In Ampersand Syntax:

\begin{verbatim}
CLASSIFY Task ISA Event
\end{verbatim}

In Ampersand's implementation, Ampersand does not require writing the types when writing Invariants.
We can consider expressing the same invariant as earlier without writing types: $\mathrm{before}\typesemi \mathrm{scheduled} \typesubset \mathrm{scheduled}\typesemi \mathrm{before}$.
If we require that invariants are well-typed, there is only one way of annotating such expressions with types.
The use of the possibly ambiguous relation name $\mathrm{before}$ is allowed as long as it is clear which relation is intended, as is the case here.
We refer to the problem of determining which relation is intended as the disambiguation problem.%, which we investigate more closely in a separate section.
%
%\subsection{Generalization}
%A common task in semantic modeling is to identify that some concepts are generalizations or specializations of other concepts.
%This section motivates the need for a special language feature to deal with generalizations.
%
%As an example, our planning application allows regular working hours to be set, which means that some times fall during regular hours.
%This means that such times would be a specialization of a time.
%We can model this with a relation called $\mathrm{worktime}$ and an invariant: $\mathrm{worktime} \subseteq \mathbb{1}_{Time}$, where $\mathbb{1}_{A} = \{(a,a) \mid a\in A \}$ is the relational identity.
%Another way to model this, is to introduce a concept $\mathrm{Worktime}$ and use $\mathbb{1}_\mathrm{Worktime}$ in the place of the relation $\mathrm{worktime}$, along with the invariant $\mathbb{1}_\mathrm{Worktime} \subseteq \mathbb{1}_{Time}$.
%However, this latter invariant is ill-typed: the expression on the left of the subset ($\subseteq$) is of type $(\mathrm{Worktime},\mathrm{Worktime})$, while the expression on the right is of type $(\mathrm{Time},\mathrm{Time})$.
%To allow expressions like $\mathbb{1}_\mathrm{Worktime} \subseteq \mathbb{1}_{Time}$ in a programming language, we need a more refined notion of which expressions are `type correct'.
%A paper by van der Woude and Joosten shows how one can do this~\cite{Woude11}.
%This paper has largely the same notion of type correctness.
%However, that paper does not consider the interplay with disambiguation which we consider here.
%
%\subsection{Two interpretations.}
%There are two different ways in which we can think of the two binary relations called $\mathrm{before}$.
%In one interpretation, we can say that $\mathrm{before}$ is a single binary relation, defined over the (disjoint) union between Task and Time.
%Whenever we use $\mathrm{before}$, we restrict its type to either \verb=Task * Task= or \verb=Time * Time= to get two completely separate behaviors.
%This means that we read $\mathrm{before}_{\mathrm{Task * Task}}$ as $\mathrm{before} \cap (\mathrm{Task}\times\mathrm{Task})$.
%
%Another interpretation is to say that there are two entirely separate relations, $\mathrm{before}_{\mathrm{Task * Task}}$ and $\mathrm{before}_{\mathrm{Time * Time}}$.
%In this interpretation, we can consider ${\mathrm{Task * Task}}$ as a part of the name of the relation.
%
%Note our interpretation does not matter for the behavior we get from well-typed invariants like the one we saw before.
%However, a problem arises if we add generalization.
%
%If we were to state - for some obscure reason - that every Task is a Time, in our first interpretation we would expect that $\mathrm{before}_{\mathrm{Task * Task}} = \mathrm{before}_{\mathrm{Time * Time}} \cap (\mathrm{Task}\times\mathrm{Task})$, since $(\mathrm{Task}\times\mathrm{Task}) \subseteq (\mathrm{Time}\times\mathrm{Time})$.
%In our second interpretation, we would not expect the behavior to change.
%
%Observe that we are free to add $\mathrm{before}_{\mathrm{Task * Task}} = \mathrm{before}_{\mathrm{Time * Time}} \cap (\mathrm{Task}\times\mathrm{Task})$ as an invariant under the second interpretation, mimicking the behavior of the first interpretation under the second interpretation.
%Therefore, we conclude that working with the second interpretation is most flexible and propose to follow this interpretation.

%
%
% @Bas, ik lees dat als:
%\begin{eqnarray}
%    \pair{a}{b}\in r&\Rightarrow&a\in\mathrm{Time}\cup\mathrm{Task}\wedge b\in\mathrm{Time}\cup\mathrm{Task}\\
%    \mathrm{before}_{\mathrm{Time * Time}}&=&r\cap (\mathrm{Time}\times\mathrm{Time})\\
%    \mathrm{before}_{\mathrm{Task * Task}}&=&r\cap (\mathrm{Task}\times\mathrm{Task})
%\end{eqnarray}
%Another interpretation is to say that there are two entirely separate relations,
%$r:\mathrm{before}_{\mathrm{Time * Time}}$ and $r':\mathrm{before}_{\mathrm{Task * Task}}$.
%In this interpretation, we can consider ${\mathrm{Task * Task}}$ as a part of the name of the relation.
%
%Bas, ik lees dat als:
%\begin{eqnarray}
%    \pair{a}{b}\in r&\Rightarrow&a\in\mathrm{Time}\wedge b\in\mathrm{Time}\\
%    \pair{a}{b}\in r'&\Rightarrow&a\in\mathrm{Task}\wedge b\in\mathrm{Task}\\
%    \mathrm{before}_{\mathrm{Time * Time}}&=&r\\
%    \mathrm{before}_{\mathrm{Task * Task}}&=&r'
%\end{eqnarray}


\section{Solving the disambiguation problem without generalization}
As an aside, we ask whether a compiler can be expected to figure out which relations are intended in reasonable time. In this section, we answer this question for a very simple setting in the affirmative by giving a simple efficient algorithm.

We can define the disambiguation problem as follows:

A disambiguation problem is given by an expression $E$ using the operations $\cup$ and $;$, and relation names from a set of names $\mathcal{N}$, and a set of possible types of each name: $\mathcal{t}(n) \subseteq \mathcal{T}\times \mathcal{T}$ for $n\in \mathcal{N}$.
A solution to the disambiguation problem is a well-typed expression $S$ in which the annotations to relation names corresponds to an element in $\mathcal{t}(n)$, such that dropping the types from the expression yields $E$.

As an example, $E = (\mathrm{before};\mathrm{scheduled}) \cup (\mathrm{scheduled};\mathrm{before})$ with types $\mathcal{t}(\mathrm{before}) = \{(\mathrm{Task}, \mathrm{Task}),(\mathrm{Time}, \mathrm{Time})\}$ and $\mathcal{t}(\mathrm{scheduled}) = \{(\mathrm{Task}, \mathrm{Time})\}$, is a disambiguation problem.
It's (here unique) solution would be:
\[\left(\mathrm{before}_{\mathrm{Task * Task}};\mathrm{scheduled}_{\mathrm{Task * Time}}\right) \cup \left(\mathrm{scheduled}_{\mathrm{Task * Time}};\mathrm{before}_{\mathrm{Time * Time}}\right)\]

We can solve the disambiguation problem using two passes through $E$.
In the first pass, we decorate the expression tree for $E$ with all the possible types for each sub-expression bottom up.
Let $p(e)$ denote the possible types, then $p(n) = \mathcal{t}(n)$, and $p(e_1;e_2) = \{(x,y) \mid \exists z. (x,z) \in p(e_1) \wedge (z,y) \in p(e_2))\}$, and $p(e_1 \cup e_2) = p(e_1) \cap p(e_2)$.
If $p(E)$ is the empty set, then there is no solution to the disambiguation problem, and we are done.
If $p(E)$ contains multiple elements, then there are multiple solutions to the disambiguation problem.
We can arbitrarily select a single element from $p(E)$ to get an arbitrary solution\footnote{In practice, we might want to indicate that $E$ is ambiguous, which could require running the next step twice: once for two different elements in $p(E)$.}.

% @Bas, deze paragraaf is wellicht leesbaarder als je gewoon de functie à la Haskell uitschrijft.
For the top-down step, let $d(e,t)$ denote the disambiguated expression $e$ with the type $t$.
For the top-level call of $d(e,t)$, let $e=E$ and $t\in p(E)$.
Then $d(n,t) = n_{[t]}$, $d(e_1 \cup e_2, t) = d(e_1,t) \cup d(e_2,t)$ and $d(e_1 ; e_2, (x,y)) = d(e_1, (x,z)) ; d(e_2, (z,y))$ where $z$ is picked such that $(x,z)\in p(e_1)$ and $(z,y)\in p(e_2)$.
Such a $z$ is guaranteed to exist by how $p$ is defined.

As an example, let's look at $E = (\mathrm{before};\mathrm{scheduled}) \cup (\mathrm{scheduled};\mathrm{before})$.
Then $p(\mathrm{before}) =  \{(\mathrm{Task}, \mathrm{Task}),(\mathrm{Time}, \mathrm{Time})\}$ following $\mathcal{t}$ and similarly $p(\mathrm{scheduled}) = \{(\mathrm{Task}, \mathrm{Time})\}$.
This gives subterms $p(\mathrm{before};\mathrm{scheduled}) = p(\mathrm{scheduled};\mathrm{before}) = \{(\mathrm{Task}, \mathrm{Time})\}$ so $p(E) = \{(\mathrm{Task}, \mathrm{Time})\}$.

Top-down, $d(E,t) = d(\mathrm{before};\mathrm{scheduled},t) \cup d(\mathrm{scheduled};\mathrm{before},t)$ with $t=(\mathrm{Task}, \mathrm{Time})$.
Since $E$ is not ambiguous, we have only one possible way to continue as we solve the disambiguation problem:
\begin{align*}d(\mathrm{before};\mathrm{scheduled},t) &= d(\mathrm{before},(\mathrm{Task}, \mathrm{Task})) ;  d(\mathrm{scheduled},(\mathrm{Task}, \mathrm{Time})) \\
&=\mathrm{before}_{\mathrm{Task} * \mathrm{Task}}; \mathrm{scheduled}_{\mathrm{Task}* \mathrm{Time}}\\
d(\mathrm{scheduled};\mathrm{before},t) &=  d(\mathrm{scheduled},(\mathrm{Task}, \mathrm{Time}));d(\mathrm{before},(\mathrm{Time}, \mathrm{Time}))\\
&=\mathrm{scheduled}_{\mathrm{Task} * \mathrm{Time}}; \mathrm{before}_{\mathrm{Time}* \mathrm{Time}}\\
\end{align*}

The disambiguation problem and the corresponding algorithm can be extended over all operations in relation algebra.
It can even be used to disambiguate special relations like the identity relation.
Let's take $E_2 = \mathrm{scheduled};(\mathrm{before} \cup \mathbb{1})$ as an expression.
The types are as in the previous example, with $\mathcal{t}(\mathbb{1}) = \left\{(t,t)\mid t\in \mathcal{T}\right\}$.
This gives $p(E_2) = \{\mathrm{Task} * \mathrm{Time}\}$ and $d(\{\mathrm{Task} * \mathrm{Time}\},E_2) = \mathrm{scheduled}_{\mathrm{Task} * \mathrm{Time}};(\mathrm{before}_{\mathrm{Time} * \mathrm{Time}}\ \cup\ \mathbb{1}_{\mathrm{Time}})$, as desired.

\section{Solving Disambiguation with Generalization}
Before giving a solution to this % @Bas, een referentie over de grens van een paragraaf is taalkundig al niet fraai, maar over de grens van een section is niet slim. Als lezer wil ik iets meer duidelijkheid dan "this" als het gaat over welk probleem je naar verwijst.
problem, we first observe that the disambiguation algorithm discussed earlier fails for our running example.
Following the declarations of the script, we let $\mathcal{t}(\mathrm{before}) = \{(\mathrm{Task}, \mathrm{Task}),(\mathrm{Time}, \mathrm{Time})\}$ and $\mathcal{t}(\mathrm{scheduled}) = \{(\mathrm{Event}, \mathrm{Time})\}$.
Now the above algorithm would conclude with an error on $E = (\mathrm{before};\mathrm{scheduled}) \cup (\mathrm{scheduled};\mathrm{before})$.

The paper by van der Woude and Joosten~\cite{Woude11} introduced an embedding relation $\epsilon_{A*B} : A \times B$ to write sentences in heterogeneous relation algebra with sub-typing in terms of heterogeneous relation algebra without it.
It can be thought of as defined by $\epsilon_{A*B} = \{(a,a) \mid a\in A \wedge a\in B\}$: it is the largest subset of the identity relation that fits the type $A\times B$.

This paper follows a similar approach: We define a language to be used by the Ampersand programmer that allows writing relations without their types as well as subtyping.
Next, we define a language that is interpreted as a heterogeneous relation algebra with a typical set-interpretation.
% @Bas, de volgende zin snap ik niet...
The semantics of the language with subtyping will be determined by what the translation process gives us, but we validate the translation by inspecting certain examples and examining the validity of Tarski's axioms for this language under the resulting interpretation.
%We deviate slightly from the ideas by van der Woude and Joosten: Their $\epsilon$ relation is a subtyping relation, so $\epsilon_{A*B}$ would exist only if $A$ is a subtype of $B$.
%In this paper, we relax that requirement.

\subsection{Language definitions}
In this section, we introduce the language in which an Ampersand programmer can write terms, as well as the language for which the disambiguation problem is considered `solved'.
The intended interpretation of these symbols is that of the regular relation algebraic operations, but we delay talking about semantics until later in this paper.

Let $\mathcal{S}$ be the language of heterogeneous relation algebra with subtyping.
It is defined by the following grammar, where $n\in\mathcal{N}$ and $A,B\in\mathcal{T}$:
\[
\varphi,\psi\in\mathcal{S} \mathbin{:=} n \mid \typetyped{\varphi}{A*B} \mid \typevee \mid \typeident \mid \typenil \mid \varphi \typeunion \psi \mid \varphi \typeinter \psi \mid \varphi \typesemi \psi \mid \typecomp{\psi}\mid \typeconv{\psi}
\]
The grammar $\mathcal{S}$ allows the programmer to write relations with their type, $\typetyped{n}{A*B}$ and $\typetyped{\typevee}{A*B}$, as usual.
However, the grammar is slightly more general, allowing type annotations for arbitrary terms $\typetyped{\varphi}{A*B}$.

Our aim is to disambiguate expressions written in this language into `normal' heterogeneous relation algebra, that is: without subtyping.
To signal certain type errors, we add an additional element $\bot$ to the set of concepts, that we will assume is different from every element in $\mathcal{T}$.
Since $\mathcal{T}$ is referred to as the set of concepts, we will refer to $\mathcal{T}\cup\{\bot\}$ as the set of pseudo-concepts.
We use the following grammar to describe the language $\mathcal{R}$ for that, where $n_{A*B}$ satisfies $(A,B)\in \mathcal{t}(n)$, and $A,B\in\mathcal{T}\cup\{\bot\}$ for all other occurrences of $A$ and $B$:
\[
\varphi,\psi\in\mathcal{R} \mathbin{:=} n_{A*B} \mid \epsilon_{A*B} \mid T_{A*B} \mid 1_{A} \mid F_{A*B} \mid \varphi \cup \psi \mid \varphi \cap \psi \mid \varphi ; \psi \mid \overline{\psi}\mid \conv{\psi}
\]

The type of a term in $\mathcal{R}$ is given by a partial typing function, which we define inductively via the relation symbol $:$ as follows:
\begin{align*}
x_{A*B} &: A * B & \text{For $x \in \mathcal{N} \cup \{\epsilon, T, F\}$ and $A,B\in \mathcal{T}$}\\
1_{A} &: A * A&\\
\varphi \odot \psi &: A * B &\quad\text{For $\odot \in \{\cup,\cap\}$, $\varphi : A * B$ and $\psi : A * B$}\\
\varphi ; \psi &: A * B &\text{For $\varphi : A * C$ and $\psi : C * B$}\\
\overline{\psi} &: A * B &\text{For $\psi : A * B$}\\
\conv{\psi} &: A * B &\text{For $\psi : B * A$}
\end{align*}
We say that a term $r\in \mathcal{R}$ is \emph{well-typed} if $r : A*B$ for some $A$ and $B$.
By induction on the inductive definition of $\mathcal{R}$, the relation $:$ is between $\mathcal{R}$ and types $\mathcal{T}\times\mathcal{T}$ and univalent.
In other words: well-typed terms in $\mathcal{R}$ have precisely one type, all of its sub-terms are well-typed, and $\bot$ does not occur as a pseudo-concept.

\subsection{Translation process}
Rather than translating terms in $\mathcal{S}$ into $\mathcal{R}$ directly, we create an intermediate language $\mathcal{D}$, which contains $\mathcal{R}$ (that is: $\mathcal{R} \subseteq \mathcal{D}$).
This will allow us to use a purely syntactic translation from $\mathcal{S}$ to $\mathcal{D}$, after which we run a disambiguation algorithm to turn it into a term in $\mathcal{R}$.
The language $\mathcal{D}$ is defined as follows, where $\xi \in \mathcal{R}$, $\Xi \subseteq \mathcal{R}$, and $A,B\in\mathcal{T}$:
\[
\varphi,\psi\in\mathcal{D} \mathbin{:=} \xi \mid \varphi \cup \psi \mid \varphi \cap \psi \mid \varphi ; \psi \mid \overline{\psi}\mid \conv{\psi} \mid \typetyped{\varphi}{A*B} \mid \Xi
\]

The term $\Xi$ is used to indicate that elements in $\Xi$ are terms that might be filled in on this position.
For example, the ambiguous term $\mathrm{before}$ in $\mathcal{S}$ would be represented in $\mathcal{D}$ as $\{\mathrm{before}_{Task*Task},\mathrm{before}_{Time*Time}\}$.
We syntactically distinguish terms of the shape $\Xi$ by writing set brackets $\{\}$ around them.
Type annotations from $\mathcal{S}$ are kept in $\mathcal{D}$, but the structure is otherwise the same as that of $\mathcal{R}$.

The translation from $\mathcal{S}$ to $\mathcal{D}$ will depend on the users's \verb=CLASSIFY= statements.
We write $A\preceq B$ to indicate that the pair $A,B$ is in the transitive closure of the pairs written \verb=CLASSIFY _ ISA _=, meaning that all elements in $A$ must occur in $B$ whenever $A\preceq B$.
Moreover, we let $\bot$ be a least element in this preorder: $\bot \preceq A$ for all $A$.
Since there are no classify statements containing $\bot$ (the user cannot write them), the least element in $\preceq$ is unique.

The translation $s : \mathcal{S} \to \mathcal{D}$ is as follows:
\begin{align*}
s(n) &= \{n_{A*B} \mid (A,B)\in\mathcal{t}(n)\}\\
s(\typetyped{\varphi}{A*B}) &= \typetyped{s(\varphi)}{A*B}\\
s(\typevee) &= \{T_{A*B} \mid A,B\in\mathcal{T}\}\\
s(\typenil) &= \{F_{A*B} \mid A,B\in\mathcal{T}\}\\
s(\typeident) &= \{1_{A} \mid A\in\mathcal{T}\}\\
s(\varphi \typeunion \psi) &= \conv{\varepsilon};s(\varphi);\varepsilon\ \cup\ \conv{\varepsilon};s(\psi);\varepsilon&\quad\text{with $\varepsilon = \{\epsilon_{A*B} \mid A\preceq B \}$} \\
s(\varphi \typeinter \psi) &= \conv{\varepsilon};s(\varphi);\varepsilon\ \cap\ \conv{\varepsilon};s(\psi);\varepsilon&\quad\text{with $\varepsilon = \{\epsilon_{A*B} \mid B\preceq A \}$}\\
s(\varphi \typesemi \psi) &= s(\varphi);\varepsilon;s(\psi)&\text{with $\varepsilon = \{\epsilon_{A*B} \mid B\preceq A \vee A\preceq B \}$}\\
s(\typecomp{\varphi}) &= \overline{s(\varphi)}\\
s(\typeconv{\varphi}) &= \conv{s(\varphi)}
\end{align*}


\paragraph{[Aside],} The Ampersand language also allows for \verb=CLASSIFY A IS B /\ C= statements, meaning that elements are in $A$ if and only if they are in $B$ and in $C$.
The information about this captured in $\preceq$ is just that $A \preceq B$ and $A\preceq C$, which means that information is lost.
We can consider the lost information when dealing with the $\typesemi$ operation by changing the $\epsilon$ relations introduced there.
However, it is an open problem to deal with this with the $\typeinter$ operation, so we give a simpler presentation that ignores this additional information in \verb=CLASSIFY _ IS _= statements for now.



\section{Bibliography}
\bibliographystyle{elsarticle-harv}
\bibliography{Disambiguation}

% van der Woude paper: https://link.springer.com/chapter/10.1007/978-3-642-21070-9_25
\end{document}
