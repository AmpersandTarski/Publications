\documentclass{elsarticle}
\usepackage{graphicx}
%\usepackage{multicol}
%\usepackage{footmisc}
\usepackage{amstext}
\usepackage{amsmath}
\usepackage{amssymb}
\usepackage{amsthm}
\usepackage[english]{babel}
%\usepackage[official,right]{eurosym}
\selectlanguage{english}
\hyphenation{ExecEngine}
\newtheorem{lemma}{Lemma}
\def\Events{{\mathit E}}
\begin{document}
% Ampersand -----------------------------------------------------------

\def\id#1{\text{\em #1\/}}
\newcommand{\code}[1]{\text{\tt\small #1}}
\newcommand{\stmtText}[1]{``{\small\tt #1}''}
\newcommand{\dom}[1]{\id{dom}(#1)}
\newcommand{\cod}[1]{\id{cod}(#1)}
\renewcommand{\int}[2]{\id{inter}(#1,#2)}
\newcommand{\pop}[1]{\id{pop}(#1)}
\newcommand{\src}[1]{\id{src}(#1)}
\newcommand{\trg}[1]{\id{trg}(#1)}
\newcommand{\powerset}[1]{\cal{P}\{#1\}}
\newcommand{\theCode}{\url{http://cs.ru.nl/~B.Joosten/ampTypes/}}
\newcommand{\la}{\langle}
\newcommand{\ra}{\rangle}
\newcommand{\full}{V}
\newcommand{\declare}[3]{\id{#1}_{\pair{\id{\small #2}}{\id{\small #3}}}}
\newcommand{\fullt}[2]{V_{\pair{#1}{#2}}}
\newcommand{\iden}[1]{I}
\newcommand{\ident}[1]{I_{\id{\small #1}}}
\newcommand{\expr}[3]{(#1)_{#2\times #3}}
\newcommand{\pair}[2]{\la{#1},{#2}\ra}
\newcommand{\atom}[1]{{\tt\small #1}}
\newcommand{\atoms}{\mathcal{A}}
\newcommand{\concepts}{\mathcal{C}}
\newcommand{\decls}{\mathcal{D}}  %% names of relations
\newcommand{\rels}{\mathcal{R}}   %% all relations
\newcommand{\relations}{\mathcal{M}} % representing terms. M is a subset of R.
\newcommand{\terms}{\mathcal{T}}
\newcommand{\vertices}{N}
\newcommand{\rules}{\mathcal{H}}
\newcommand{\tf}[1]{\mathfrak{T}(#1)}
\newcommand{\ptf}[1]{\mathfrak{T}'(#1)}
\newcommand{\ti}[1]{\mathfrak{I}(#1)}
\newcommand{\tic}[1]{I_{\cal C}(#1)}
\newcommand{\relAdd}{\dagger}
\newcommand{\flip}[1]{{#1}^\smallsmile} %formerly:  {#1}^\backsim
\newcommand{\kleeneplus}[1]{{#1}^+}
\newcommand{\kleenestar}[1]{{#1}^*}
\newcommand{\cmpl}[1]{\overline{#1}}
\newcommand{\rel}{\times}
\newcommand{\compose}{;}
\newcommand{\subs}{\subseteq}%{\models}
\newcommand{\fun}{\rightarrow}
\newcommand{\isa}{\preceq}
%\newcommand{\isaClos}{\sqsubseteq}
\newcommand{\typetest}{?}
\newcommand{\meet}{\sqcap}
\newcommand{\join}{\sqcup}
\newcommand{\Meet}{\bigsqcap}
\newcommand{\Moin}{\bigsqcup} % because LaTeX has already defined command \Join.
\newcommand{\order}{\ominus}
\newcommand{\anything}{\top}
\newcommand{\nothing}{\bot}
\newcommand{\rewriteto}{\rightarrow}
\newcommand{\calc}{\implies}
\newcommand{\alland}{\bigwedge}
\newcommand{\mph}[3]{#1_{#2\times #3}}
\newcommand{\mphu}[1]{#1_{\univ\times\univ}}

%-----------------------------------------
\newcommand{\kse}{\hspace*{1.7em}}
\newcommand{\ksf}{\hspace*{1em}}
\newcommand{\ksg}{\hspace*{1em}}
\newenvironment{derivation}{\begin{tabbing}\kse \= \ksf \= \ksg \= \kill}{\end{tabbing}}
\newtheorem{definition}{Definition}
\newcommand{\term}[1]{\>\>\(#1\)\\[1ex]}
\newcommand{\rela}[2]{\>\(#1\)\>\>\{ \ #2 \ \}\\[1ex]}
\newcommand{\weg}[1]{}

\def\define#1{\label{dfn:#1}{\em #1}\index{#1}}
\def\definem#1{\label{dfn:#1}{\em #1}\index{#1}\marge{#1}}
\newcommand{\marg}[1]{\index{#1}\marge{#1}}


\title{A Theory for Migration of Information Systems}
\author[ou,ordina]{Stef Joosten\fnref{fn1}}
\ead{stef.joosten@ou.nl}
\author[umn]{Sebastiaan Joosten\fnref{fn2}}
\address[ou]{Open Universiteit Nederland, Heerlen, the Netherlands}
\address[ordina]{Ordina NV, Nieuwegein, the Netherlands}
\address[umn]{University of Minnesota, Minneapolis, USA}
\fntext[fn1]{ORCID 0000-0001-8308-0189}
\fntext[fn2]{ORCID 0000-0002-6590-6220}

\begin{abstract}
	The Ampersand project has provided the theory and tools to generate semantic information systems from an algebraic specification.
	However, information systems in practice may change repeatedly after their maiden deployment.
	Changes that affect the data model typically result in a data migration.
	In such cases, simply regenerating the system is not enough.

	In this contribution we develop a theory for data migration that aims at automating the migration,
	to enable more frequent migrations.
	A theory is needed to do this reliably, without mistakes.
	This paper proposes such a theory.

	The ultimate target is to generate the entire migration from two specifications: the as-is specification and the to-be specification.
	The theory for doing that is the topic of this paper.
	The software generator that embodies this theory is subject of future research.
\end{abstract}

\begin{keyword}
relation algebra\sep software development\sep data migration\sep software migration\sep Ampersand
\end{keyword}
\maketitle

\section{Introduction}
\label{sct:Introduction}

\section{Basics}
\label{sct:Basics}
	In this section we explain the basic notions to describe migration.
	We introduce the notion of dataset.
	A dataset is a quintuple $\la\rules,\rels,\concepts,\pairs,\atoms\ra$,
	which consists of a set of rules $\rules$, a set of relations $\rels$, a set of concepts $\concepts$, a set of pairs $\pairs$, and a set of atoms $\atoms$.
	Let us introduce these five notions one by one.
\subsection{Atoms}
	Atoms (from set $\atoms$) are values that have no internal structure, meant to represent data elements in a database.
	From a business perspective, atoms are used to represent concrete items of the world,
	such as \atom{Peter}, \atom{1}, or \atom{the king of France}.
	By convention throughout the remainder of this paper, variables $a$, $b$, and $c$ are used to represent \emph{atoms}.
	The set of all atoms is called $\atoms$.
    Each atom is an instance of a \emph{concept}.
\subsection{Concepts}
	Concepts (from set $\concepts$) are names we use to classify atoms in a meaningful way.
	For example, you might choose to classify \atom{Peter} as a \concept{Person}, and \atom{074238991} as a \concept{TelephoneNumber}.
	In this example, \concept{Person} and \concept{TelephoneNumber} are concepts.
    We will use variables $A$, $B$, $C$, $D$ to represent concepts.
	The term $\ident{A}$ represents the \emph{identity relation} of concept $A$.
	The expression $a \in A$ means that atom $a$ is an \emph{instance} of concept $A$.
	The statement $A\isa B$ (pronounce: $A$ is a $B$) states that any instance of $A$ is an instance of $B$ as well.
	We call this {\em specialization}, but it is also known as {\em generalization} or {\em subtyping}.
	Specialization is needed to allow statements such as: ``A woman is a human.'' or ``A human is a mammal.''.
\subsection{Pairs}
	Pairs (from set $\pairs$) are used to relate two atoms.
	A pair $\pair{\text{\atom{Peter}}}{\text{\atom{074238991}}}$ might bear the meaning that the ``thing'' that \atom{Peter} refers to
	has \atom{074238991} as a telephone number.
	In the formal world, we do not care about that meaning,
	leaving it entirely up to a user which meaning (s)he attaches to this pair.
\subsection{Relations}
	Relations (from set $\rels$) are used in information systems to store data as pairs.
	In this paper relations are represented by variables $r$, $s$, and $d$.
	Every relation $r$ has a set of pairs, $\pop{r}$, which we call the population of $r$.
	Every relation has a source concept and a target concept.
	If we need to be specific about the source and target concepts, we write $\declare{r}{A}{B}$ instead of just $r$.
	The pair $\pair{A}{B}$ is called the \emph{signature} of the relation.
	The term $\fullt{A}{B}$ represents the \emph{universal relation} over concepts $A$ and $B$.
	The term $\ident{A}$ represents the \emph{identity relation} over concept $A$.

	The signature rule guarantees that each pair in $r$ respects the signature of $r$:
\begin{equation}
	\tag{signature rule}
	\pair{a}{b}\in\pop{\declare{r}{A}{B}} \Rightarrow\ a \in A \wedge b \in B \label{typing of relations}
\end{equation}

	The population rule defines the population of a dataset:
\begin{equation}
	\tag{population rule}
	\pairs\ =\ \{ \pair{a}{b}|\ r\in\rels,\ \pair{a}{b}\in\pop{r}\} \label{def. population}
\end{equation}

	The atoms rule defines all atoms in a dataset:
\begin{equation}
	\tag{atoms rule}
	\atoms\ =\ \{ a |\ \pair{b}{c}\in\pairs,\ a\in\{b,c\}\ \} \label{def atoms}
\end{equation}
\subsection{Terms}
	A term is a combination of relations (from $\rels$), which we use to express constraints from the real world.
	For this purpose we use the operators from relation algebra~\cite{BrinkKahlSchmidt1997}.
\begin{definition}[terms]
\label{def:terms}
\item   The set of terms, $\terms$, is the smallest set that satisfies, for all $r,s \in \terms$, $d\in\rels$ and $A,B \in \concepts$: 
\begin{eqnarray*}
	d&\in&\terms         \quad\quad\text{(every relation is a term)}\\
	(r \cup s)&\in&\terms\quad\quad\text{(union)}\\
	(r \cap s)&\in&\terms\quad\quad\text{(intersection)}\\
	(r-s)&\in&\terms     \quad\quad\text{(difference)}\label{def:difference}\\
	(r;s)&\in&\terms     \quad\quad\text{(composition)}\\
	(r\backslash s)&\in&\terms     \quad\quad\text{(right residual)}\\
	(r\slash s)&\in&\terms     \quad\quad\text{(left residual)}\\
	\flip r&\in&\terms   \quad\quad\text{(converse)}\\
	\cmpl{r}&\in&\terms   \quad\quad\text{(complement)}\\
%	\kleeneplus r&\in&\terms   \quad\quad\text{(Kleene closure)}\\
%	\kleenestar r&\in&\terms   \quad\quad\text{(Kleene closure)}\\
	\ident{A}&\in&\terms \quad\quad\text{(identity)}\\
	\fullt{A}{B}&\in&\terms \quad\quad\text{(full set)}
\end{eqnarray*}
\end{definition}
	Throughout the remainder of this paper,	terms are represented by variables $r$, $s$, $d$, and $t$.

	The relations used in a term $t$ are given by $\relsIn{t}$:
\begin{definition}[relations in terms]
	\label{relations in terms}
	\item   For every $A,B\in\concepts$ and $r,s\in\terms$
	\begin{eqnarray*}
		\relsIn{r}		 &=&\{r\}	\\
		\relsIn{r \cup s}	 &=&\relsIn{r}\cup\relsIn{s}\\
		\relsIn{r \cap s}	 &=&\relsIn{r}\cup\relsIn{s}\\
		\relsIn{r-s}	 &=&\relsIn{r}\cup\relsIn{s}\\
		\relsIn{r;s}	 &=&\relsIn{r}\cup\relsIn{s}\\
		\relsIn{r\backslash s}	 &=&\relsIn{r}\cup\relsIn{s}\\
		\relsIn{r\slash s}	 &=&\relsIn{r}\cup\relsIn{s}\\
		\relsIn{\flip{r}}	 &=&\relsIn{r}\\
		\relsIn{\cmpl{\declare{r}{A}{B}}}	 &=&\relsIn{r}\\
	%	\relsIn{\kleeneplus{r}}	 &=&\relsIn{r}\\
	%	\relsIn{\kleenestar{r}}	 &=&\relsIn{r}\\
		\relsIn{\ident{A}} 	 &=&\emptyset\\
		\relsIn{\fullt{A}{B}}&=&\emptyset\\
	\end{eqnarray*}
\end{definition}
	The \define{signature} of a term $r$ is a pair of concepts, given by $\tf{r}$.
	$\mathfrak{T}$ is a function that maps terms to signatures.
	This function and its peculiarities are discussed in~\cite{Joosten2015}
	and have no consequences for the remainder of this paper.
	(\emph{\@Bas, komen we ermee weg als we de definitie van $\tf{r}$ hier weglaten?})

	Every term $t$ has a set of pairs $\pop{t}$, which we call the population of that term.
 	Let $A$ and $B$ be finite sets of atoms, then the population of term is defined as follows:
\begin{definition}[population of terms]
\label{population of terms}
\item   For every $A,B\in\concepts$ and $r,s\in\terms$
\begin{eqnarray*}
	\pop{r}		 &=&\{\pair{a}{b}|\ a\ r\ b\}	\\
	\pop{r \cup s}	 &=&\{\pair{a}{b}|\ \pair{a}{b}\in\pop{r}\ \text{or }\ \pair{a}{b}\in\pop{s}\}	\\
	\pop{r \cap s}	 &=&\{\pair{a}{b}|\ \pair{a}{b}\in\pop{r}\ \text{and}\ \pair{a}{b}\in\pop{s}\}	\\
	\pop{r-s}	 &=&\{\pair{a}{b}|\ \pair{a}{b}\in\pop{r}\ \text{and}\ \pair{a}{b}\notin\pop{s}\}	\\
	\pop{r;s}	 &=&\{\pair{a}{c}|\ \text{for some}\ b,\ \pair{a}{b}\in\pop{r}\ \text{and}\ \pair{b}{c}\in\pop{s}\}	\\
	\pop{r\backslash s}	 &=&\{\pair{b}{c}|\ \text{for all}\ a,\ \pair{a}{b}\in\pop{r}\ \text{implies}\ \pair{a}{c}\in\pop{s}\}	\\
	\pop{r\slash s}	 &=&\{\pair{a}{b}|\ \text{for all}\ c,\ \pair{b}{c}\in\pop{s}\ \text{implies}\ \pair{a}{c}\in\pop{r}\}	\\
	\pop{\flip{r}}	 &=&\{\pair{b}{a}|\ \pair{a}{b}\in\pop{r}    \}	\\
	\pop{\cmpl{\declare{r}{A}{B}}}	 &=&\fullt{A}{B}-r	\\
%	\pop{\kleeneplus{r}}	 &=&\text{the smallest set that satisfies }\kleeneplus{r}=r\ \cup\ \kleeneplus{r};\kleeneplus{r}	\\
%	\pop{\kleenestar{\declare{r}{A}{A}}}	 &=&\ident{A}\cup\kleeneplus{r}\\
	\pop{\ident{A}} 	 &=&\{\pair{a}{a}|\ a\in A\}	\\
	\pop{\fullt{A}{B}}&=&\{\pair{a}{b}|\ a\in A, b\in B\}
\end{eqnarray*}
\end{definition}

\subsection{Rules}
	A rule (from set $\rules$) is an equality between two terms that constrains the data stored in relations,
	to ensure that all rules in the dataset are satisfied.
	A \define{rule} is a pair of terms $r,s\in\terms$ with the same types ($\tf{r}=\tf{s}$), which we denote as:
\[\text{RULE}\ r = s\]
	This means that the constraint \(\pop{r} = \pop{s}\) is satisfied throughout the dataset.
	In practice, we will also use \(\text{RULE}\ r\subs s\) as a shorthand for \(\text{RULE}\ r\cap s = r\).

	We call a rule $\text{RULE}\ r = s$ \define{satisfied} if \(\pop{r} = \pop{s}\).
	Each relation that is used in a term of a dataset $\la\rules,\rels,\concepts,\pairs,\atoms\ra$ is an element of $\rels$ as well:
\begin{equation}
	\tag{relations in dataset}
	\{ r|\ \text{RULE}\ t = s\in\rules,\ r\in\relsIn{t}\cup\relsIn{s}\ \}\subseteq\rels\label{def relations in dataset}
\end{equation}

	The data in a data set $\la\rules,\rels,\concepts,\pairs,\atoms\ra$ are the pairs in the population of all relations in $\rels$.
	This constitutes the set $\pairs$, also known as the population of the dataset.
	In response to messages such as
\[\text{INSERT}\ <3,{\tt Brown}> \text{INTO}\ \declare{r}{Age}{Name}\]
 	the population of a dataset changes. 
	As the population of relations used in $r$ changes with time, the satisfaction of the rule changes accordingly.
	A software developer, who conceives these rules, must consider how to keep each one of them satisfied.
	We call a rule \define{violated} if it is not satisfied.
	The set $\pop{(s-r)\cup(r-s)}$ is called the \emph{violation set} of \(\text{RULE}\ r = s\).
	To \define{resolve} violations means to change the contents of relations such that the rule is satisfied%
\footnote{To \define{restore invariance} is sometimes used as a synonym to resolving violations.
	To \define{preserve invariance} is sometimes used as a synonym to keeping a rule satisfied.}.
	Consequently, such a rule may also be referred to as \define{invariant}.
	Each pair in the violation set of a rule is called a violation of that rule.

\section{Bibliography}
\bibliographystyle{elsarticle-harv}
\bibliography{doc}


\end{document}
