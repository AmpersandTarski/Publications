\documentclass{elsarticle}
\usepackage{graphicx}
%\usepackage{multicol}
%\usepackage{footmisc}
\usepackage{amstext}
\usepackage{amsmath}
\usepackage{amssymb}
\usepackage{amsthm}
\usepackage[english]{babel}
%\usepackage[official,right]{eurosym}
\selectlanguage{english}
\include{preamramics}
\hyphenation{ExecEngine}
\newtheorem{lemma}{Lemma}
\def\id#1{\text{\it #1\/}}
\def\Events{{\mathit E}}
\begin{document}

\title{A Theory for Migration of Information Systems}
\author[ou,ordina]{Stef Joosten\fnref{fn1}}
\ead{stef.joosten@ou.nl}
\author[umn]{Sebastiaan Joosten\fnref{fn2}}
\address[ou]{Open Universiteit Nederland, Heerlen, the Netherlands}
\address[ordina]{Ordina NV, Nieuwegein, the Netherlands}
\address[umn]{University of Minnesota, Minneapolis, USA}
\fntext[fn1]{ORCID 0000-0001-8308-0189}
\fntext[fn2]{ORCID 0000-0002-6590-6220}

\begin{abstract}
	The Ampersand project has provided the theory and tools to generate semantic information systems from an algebraic specification.
	However, information systems in practice may change repeatedly after their maiden deployment.
	Changes that affect the data model typically result in a data migration.
	In such cases, simply regenerating the system is not enough.

	In this contribution we develop a theory for data migration that aims at automating the migration,
	to enable more frequent migrations.
	A theory is needed to do this reliably, without mistakes.
	This paper proposes such a theory.

	The ultimate target is to generate the entire migration from two specifications: the as-is specification and the to-be specification.
	The theory for doing that is the topic of this paper.
	The software generator that embodies this theory is subject of future research.
\end{abstract}

\begin{keyword}
relation algebra\sep software development\sep data migration\sep software migration\sep Ampersand
\end{keyword}
\maketitle

\section{Introduction}
\label{sct:Introduction}

\section{Bibliography}
\bibliographystyle{elsarticle-harv}
\bibliography{doc}


\end{document}
