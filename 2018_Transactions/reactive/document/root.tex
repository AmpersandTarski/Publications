\documentclass[11pt,a4paper]{article}
\usepackage{isabelle,isabellesym}
\usepackage{fullpage}
\usepackage[usenames,dvipsnames]{color}
\usepackage{document}

% further packages required for unusual symbols (see also
% isabellesym.sty), use only when needed

\usepackage{amssymb}
  %for \<leadsto>, \<box>, \<diamond>, \<sqsupset>, \<mho>, \<Join>,
  %\<lhd>, \<lesssim>, \<greatersim>, \<lessapprox>, \<greaterapprox>,
  %\<triangleq>, \<yen>, \<lozenge>

\usepackage[english]{babel}
  %option greek for \<euro>
  %option english (default language) for \<guillemotleft>, \<guillemotright>

\usepackage{stmaryrd}
  %for \<Sqinter>

\usepackage{eufrak}
  %for \<AA> ... \<ZZ>, \<aa> ... \<zz> (also included in amssymb)

%\usepackage{textcomp}
  %for \<onequarter>, \<onehalf>, \<threequarters>, \<degree>, \<cent>,
  %\<currency>

% this should be the last package used
\usepackage{pdfsetup}

% urls in roman style, theory text in math-similar italics
\urlstyle{rm}
\isabellestyle{it}

% for uniform font size
%\renewcommand{\isastyle}{\isastyleminor}

\begin{document}

\title{Generalised Reactive Processes in Isabelle/UTP}

\author{Simon Foster \and Samuel Canham}

\maketitle

\begin{abstract}
  Hoare and He's UTP theory of reactive processes provides a unifying foundation for the semantics
  of process calculi and reactive programming. A reactive process is a form of UTP relation which
  can refer to both state variables and also a trace history of events. In their original presentation,
  a trace was modelled solely by a discrete sequence of events. Here, we generalise the trace model 
  using ``trace algebra'', which characterises traces abstractly using cancellative monoids, and 
  thus enables application of the theory to a wider family of computational models, including 
  hybrid computation. We recast the reactive healthiness conditions in this setting, and prove
  all the associated distributivity laws. We tackle parallel composition of reactive processes
  using the ``parallel-by-merge'' scheme from UTP. We also identify the associated theory of
  ``reactive relations'', and use it to define generic reactive laws, a Hoare logic, and a weakest
  precondition calculus.
\end{abstract}

\tableofcontents

% sane default for proof documents
\parindent 0pt\parskip 0.5ex

% generated text of all theories
\input{session}

% optional bibliography
\bibliographystyle{abbrv}
\bibliography{root}

\end{document}
