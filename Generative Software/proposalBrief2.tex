\documentclass{elsarticle}
% \usepackage{graphicx}
% \usepackage{footmisc}
% \usepackage{amstext}
% \usepackage{amsmath}
% \usepackage{amssymb}
% \usepackage{amsthm}
\usepackage[english]{babel}
\usepackage{libertine}
\usepackage{inconsolata}
% \usepackage[official,right]{eurosym}
\selectlanguage{english}
\frenchspacing

\begin{document}
% \include{preamble}

\title{Enhancing Software Development with Artificial Intelligence}
\author{prof.dr.ir.\ Stef Joosten\footnote{ORCID 0000-0001-8308-0189}}
\author{dr.\ Tim Steenvoorden\footnote{ORCID 0000-0002-8436-2054}}
\address{Open Universiteit Nederland, Heerlen, the Netherlands}

\begin{abstract}
\end{abstract}

\begin{keyword}
    generative software, AI enhanced software engineering, reliability by construction, prototyping, correctness by construction, Ampersand, TopHat
\end{keyword}
\maketitle

\section{Problem area}
    Recent developments in Artificial Intelligence (e.g.~\cite{Chen2021EvaluatingLL,Ernst2022AIDrivenDI,Imai2022}) have attracted
    the attention of the software engineering community to unleash its potential for software engineering.
    This proposal addresses the problem of developing software correctly and reliably by construction,
    leveraging the advances in AI for augmented software engineering for solving an important issue of large IT-projects in practice.

    We hold it to be self-evident that correctness and reliability by construction requires software to be generated from a formal specification.
    Contemporary low-code platforms, SaaS-solutions, and model-driven development environments are examples of software generators,
    all of which require some human programming to suit the deployed artifacts to the concrete requirements from practice.
    Therefore, human errors cannot be ruled out.
    Such software generators are typically based on informal methods, so their reliability and correctness depends on testing practices.
    In contrast, formal specification languages, such as~\cite{Alloy2006,CSP,Z}, give a concrete perspective on correctness.
    However, they typically don't generate full-fledged information systems.
    As far as we know today, Ampersand~\cite{JoostenRAMiCS2017} is the only formal specification language
    that generates complete information systems from formal specification.
    Yet, formal specification languages suffer from the problem that most software engineers lack the education and skills
    to make formal specifications in real-life situations.
    These languages are typically difficult to learn and use, so they are not widely used in practice.

\section{Vision}
    Our vision consists of four beliefs:
\begin{itemize}
\item 
    We believe that formal specifications are prerequisite to the correctness and reliability of software systems.
\item 
    We believe that generating software can eliminate many opportunities for human error.
    For this reason, software generation is a neccessary ingredient of correctness and reliability by construction.
\item 
    We believe that AI can be used to make formal specification languages more accessible to software engineers,
    so they can develop correct and reliable software by construction.
\item 
    We believe that AI might cause a breakthrough in the use of formal specification languages by software engineers.
    If formal methods can be made more accessible to the software engineering community,
    we may have a game changer with a large potential for the quality of governmental, financial, and industrial data in practice.
\end{itemize}
    That summarizes our vision underlying this proposal.

    Our proposal builds on the extensive experience of the Open Universiteit with
    generating information systems~\cite{JoostenRAMiCS2017,Joosten-JLAMP2018,Steenvoorden2022,10.1145/3354166.3354182},
    combined with the AI expertise of the new research group of artificial intelligence.
    For tooling, the proposal builds exclusively on free open source software
    and the results will be available as free open source software to the public.

\section{Benefits}
    This research aims to contribute to 
\begin{itemize}
    \item Correctness by construction by working from a formal specification, enabling software engineers to use proofs besides testing.
    \item Reliability by construction by preventing human programming mistakes.
    \item Productivity of software engineers by eliminating the need to program the software.
    \item Diminished project risk, by automating software production and thus preventing a large class of mistakes.
    \item Short release cycles, so live demonstrations can keep the future user population informed and keep funders happy.
    \item Guarantees on the correctness of software, by eliciting business processes and business rules into formal specifications.
    \item Savings are expected on software development (by generating), testing (by error prevention), and maintenance of both software and tests.
\end{itemize}

\section{Researchers}
    The authors are acknowledged scientists, all with a proven track record in generative software development and workflow modelling.

\newpage

\section{Bibliography}
    \bibliographystyle{elsarticle-harv}
    \bibliography{doc}


\end{document}