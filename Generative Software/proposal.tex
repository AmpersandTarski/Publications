\documentclass{elsarticle}
% \usepackage{natbib}
% \usepackage{graphicx}
% \usepackage{footmisc}
% \usepackage{amstext}
% \usepackage{amsmath}
% \usepackage{amssymb}
% \usepackage{amsthm}
\usepackage[english]{babel}
\usepackage{libertine}
\usepackage{inconsolata}
% \usepackage[official,right]{eurosym}
\selectlanguage{english}
\usepackage{hyperref}
\frenchspacing

\begin{document}
% \include{preamble}

\title{Generative Software Development\\{\em\normalsize Research Offering}}
\author{prof.dr.ir.\ Stef Joosten\footnote{ORCID 0000-0001-8308-0189}}
\author{dr.\ Tim Steenvoorden\footnote{ORCID 0000-0002-8436-2054}}
\address{Open Universiteit Nederland, Heerlen, the Netherlands}

\begin{abstract}
    This paper summarizes the capability of the Open Universiteit in generative software development.
    We offer this enabling technology as a component of a larger effort that involves designing distributed information systems.
    Generative software development offers attractive benefits for correctness, security, and resilience.
    Also, it enables a product to be demonstrated live during its creation,
    hereby supporting agile software development methods.
    %every 2 or 3 weeks as it develops into an end product.
\end{abstract}

\begin{keyword}
    generative software, software engineering, security, prototyping, correctness, Ampersand, TopHat
\end{keyword}
\maketitle

\section{Requirements (remove this section before submitting)}
Reviewers will judge this proposal by the following requirements.
\subsection{Scientific quality}
\begin{enumerate}
   \item 	To what extent is the proposed research original and how would you rate the innovative elements?
   \item 	What is your assessment of the design of the project, including the goals, hypotheses, research methods, and scientific feasibility?
   \item 	What is your assessment of the coherence and time schedule of the proposed lines of research?
   \item 	Is the research group competent enough to carry out the research? Does the group have a relevant position in the international scientific community? Is the available infrastructure adequate?
   \item 	Are the number and category of requested personnel, budget for materials, investments, and foreign travel adequate?
   \item 	Does the proposed research fit in the Open Technology Programme; does it focus on engineering sciences research?
   \item 	What are the strong and weak points of the scientific part of the proposal?   
\end{enumerate}

\subsection{Knowledge Utilisation} (the application of the results of the research by third parties)
\begin{enumerate}
   \item  	What is your assessment of the description of the commercial and/or societal potential impacts of the research given in the proposal?
   \item 	What is your assessment of the contribution and commitment of the users and the proposed composition of the user committee?
   \item 	What is your assessment of the freedom to operate concerning commercial propositions, existing patents, eligibility or societal acceptance? 
   \item 	What are the prospects for collaboration with the industry and knowledge transfer, assuming the project is successful? Please address both aspects.
   \item 	What is your assessment of the research group's competence regarding the transfer and application of research results?
   \item 	What are the strong and weak points of the utilisation plan?   
\end{enumerate}

\section{Research question}
\label{sct:Research question}
   Data driven organizations that change frequently are facing the dilemma of buy vs. build.
   On one hand, flexible and fast adaptation requires immediate deployment of new software components, without interrupting the ongoing business.
   Commercial off-the-shelf (COTS) components can solve that problem, but they can be too generic.
   On the other hand, one-of-a-kind organizations require one-of-a-kind components that are not readily available.
   The so called ``low code an no code platforms'' are sometimes advocated as the solution
   because their (re-)sellers claim that the business itself can configure such systems.
   However, we have seen some disappointing results of that in practice, because of integration problems and maintainability issues.
   So homebrew solutions are still widespread in organizations with one-of-a-kind requirements.
   They are still willing to invest the time to build and maintain information systems and take the risk of a failing IT-innovation.
 
   Generative development can shorten the development time and reduce the development risks of homebrew software,
   regaining the desired flexibility and satisfying specific one-of-a-kind requirements at the same time.
   Current research has shown that complete working information systems can be generated.
   To get this technology ready for real use, research needs to be done on gaining more control over non-functional properties of generated systems,
   such as scalability, robustness, and security.

\section{Plan}
   We want to participate in a larger research project by means of one researcher (PhD candidate, full-time, 4 years) and one scientific programmer (professional, full-time, 4 years).
   We intend to further the knowledge about generative software development by participating in a larger research effort.
   We bring the expertise~\cite{JoostenRAMiCS2017,Steenvoorden2022} and tools~\cite{Joosten-JLAMP2018,10.1145/3354166.3354182} on generative software development
   as an asset to that project at no extra cost.
   The authors also intend to participate in person in the role of supervisors of the researcher and the scientific programmer, each for 0.1 FTE.

   This capability offers the following benefits:
\begin{itemize}
    \item By generating software, we can produce live demonstrations periodically to keep the future user population informed of the progress.
    This produces quick results, which keeps funders happy.
    \item By automating software production, we can prevent a large class of mistakes and thereby reduce the project risk.
    \item By formally specifying the functionality, we can give valuable guarantees on the correctness of the results.
    \item Software that is correctly generated needs no testing. This yields significant savings on testing and test maintenance.
    \item The tooling that has been developed at the Open University is especially suitable for automating business processes and workflows.
\end{itemize}

\section{Qualification of researchers}
\label{sct:Qualification}
   Research into generative software has a long tradition at the OU.
   Results from this research comprise:
\begin{itemize}
   \item A generator, Ampersand, that produces a complete information system from a specification,
         and a generator, TopHat, that produces a workflow system from a specification,
         demonstrating that this research has tangible results~\cite{Michels2011}.
   \item Numerous publications in peer-reviewed scientific conferences and journals,
         demonstrating that this research is scientifically relevant.
         Some of the highlights:
         \begin{itemize}
            \item A type system for information systems to ensure that all data is meaningful (i.e. has an interpretation)~\cite{vdWoude2011}
            \item A book, "Rule Based Design", published as open educational content on the web~\cite{RBD}
            \item An application that checks modeling conventions in architectural specifications~\cite{iceis22}
            \item An application for identity and access management~\cite{SIAM2008}
            \item Domain analysis~\cite{Joosten2015}.
         \end{itemize}
   \item A course, Rule Based Design, which is in the Master's program of the Computer Science curriculum and the Business Process Management \& IT program of the OU,
         demonstrating that this research is valued by the OU.
   \item over 10 generated information systems in production, made by TNO and one made by the OU,
         demonstrating that this research has practical value for industry.
   \item in-kind support from Ordina,
         demonstrating that this research can draw positive attention from the IT industry.
\end{itemize}

Current work in our group is about supporting incremental changes to an information system generatively,
to support DevOps teams who need incremental deployments.
Supporting incremental deployments with a generator is likely to decrease deployment times,
to increase the number of releases per time unit, and to decrease the risk of errors propagating to production.
\section{Bibliography}
\bibliographystyle{elsarticle-harv}
\bibliography{doc}


\end{document}