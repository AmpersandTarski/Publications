\documentclass{elsarticle}
% \usepackage{graphicx}
% \usepackage{footmisc}
% \usepackage{amstext}
% \usepackage{amsmath}
% \usepackage{amssymb}
% \usepackage{amsthm}
\usepackage[english]{babel}
% \usepackage[official,right]{eurosym}
\selectlanguage{english}
\begin{document}
% \include{preamble}

\title{Generation of user interfaces\\{\em\normalsize Research Proposal}}
\author{Stef Joosten\footnote{ORCID 0000-0001-8308-0189}}
\author{Tim Steenvoorden\footnote{ORCID 0000-0002-8436-2054}}
\address{Open Universiteit Nederland, Heerlen, the Netherlands}

\begin{abstract}
Disciplines: software engineering, 
\end{abstract}

\begin{keyword}
\end{keyword}
\maketitle

\section{Requirements}
\subsection{Scientific quality}
\begin{enumerate}
   \item 	To what extent is the proposed research original and how would you rate the innovative elements?
   \item 	What is your assessment of the design of the project, including the goals, hypotheses, research methods, and scientific feasibility?
   \item 	What is your assessment of the coherence and time schedule of the proposed lines of research?
   \item 	Is the research group competent enough to carry out the research? Does the group have a relevant position in the international scientific community? Is the available infrastructure adequate?
   \item 	Are the number and category of requested personnel, budget for materials, investments, and foreign travel adequate?
   \item 	Does the proposed research fit in the Open Technology Programme; does it focus on engineering sciences research?
   \item 	What are the strong and weak points of the scientific part of the proposal?   
\end{enumerate}

\subsection{Knowledge Utilisation} (the application of the results of the research by third parties)
\begin{enumerate}
   \item  	What is your assessment of the description of the commercial and/or societal potential impacts of the research given in the proposal?
   \item 	What is your assessment of the contribution and commitment of the users and the proposed composition of the user committee?
   \item 	What is your assessment of the freedom to operate concerning commercial propositions, existing patents, eligibility or societal acceptance? 
   \item 	What are the prospects for collaboration with the industry and knowledge transfer, assuming the project is successful? Please address both aspects.
   \item 	What is your assessment of the research group's competence regarding the transfer and application of research results?
   \item 	What are the strong and weak points of the utilisation plan?   
\end{enumerate}


\section{Research question}
   Data driven organizations that need to adapt frequently to changing circumstances face a software dilemma.
   On one hand, flexible and fast adaptation requires immediate deployment of new software components, without interrupting the ongoing business.
   On the other hand, one-of-a-kind organizations require one-of-a-kind components that need to be developed first.
   If the required component is generic enough, commercial off-the-shelf (COTS) components solve that problem.
   However, if very specific components are not on the market, organizations must revert to homebrew components.
   They either sacrifice the specific support for their business and choose a generic but readily available solution,
   or they sacrifice their flexibility and enter into a software project to achieve the specific support they desire.

   In practice, we see many organizations that opt for homebrew solutions.
   In these situations, generative software can shorten the development time and reclaim some of the desired flexibility.
   Software generators are popular for many different purposes.
   There are generators that build the database structure from a (textual or graphical) specification.
   The so called ``low code platforms'' generate user functionality from a graphical coding environment.
   When the database structure is not in-sync with the user functionality, custom code is needed to link the two together.
   Generating user interfaces
   The state of the art in generative software is that databases can be generated from specifications.
   
\label{sct:Introduction}
\section{Bibliography}
\bibliographystyle{elsarticle-harv}
\bibliography{doc}


\end{document}