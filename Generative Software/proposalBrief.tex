\documentclass{elsarticle}
% \usepackage{graphicx}
% \usepackage{footmisc}
% \usepackage{amstext}
% \usepackage{amsmath}
% \usepackage{amssymb}
% \usepackage{amsthm}
\usepackage[english]{babel}
% \usepackage[official,right]{eurosym}
\selectlanguage{english}
\begin{document}
% \include{preamble}

\title{Generative Software Development\\{\em\normalsize Research Proposal}}
\author{prof.dr.ir.\ Stef Joosten\footnote{ORCID 0000-0001-8308-0189}}
\author{dr.\ Tim Steenvoorden\footnote{ORCID 0000-0002-8436-2054}}
\address{Open Universiteit Nederland, Heerlen, the Netherlands}

\begin{abstract}
    This paper summarizes the capability of the Open Universiteit in generative software development.
    We offer this enabling technology as a component of a larger effort that involves designing distributed information systems.
    Generative software development offers attractive benefits for security and resilience,
    and enables a product to be demonstrated live every 2 or 3 weeks as it develops into an end product.
\end{abstract}

\begin{keyword}
    generative software, software engineering, security, prototyping, correctness, Ampersand, TopHat
\end{keyword}
\maketitle

\section{Problem area}
   Data driven organizations that need to adapt frequently to changing circumstances face a software dilemma.
   On one hand, flexible and fast adaptation requires immediate deployment of new software components, without interrupting the ongoing business.
   On the other hand, one-of-a-kind organizations require one-of-a-kind components that need to be developed first.
   If the required component is generic enough, commercial off-the-shelf (COTS) components or Software-as-a-Service (SaaS) components will solve that problem.
   However, if very specific components are not on the market, organizations must revert to homebrew components.
   The choice between a readily available solution and a homebrew solution can be a dilemma:
   you either sacrifice the specific support for your business or you sacrifice the required flexibility.
   In practice, we see many organizations that opt for low-code platforms to solve this dilemma.
   Triggered by the advertised prospect that business stakeholders can build the software by themselves,
   organizations can be in for a disappointment.

\section{Vision}
   Generative software development uses a software generator to build information systems from a formal specification.
   Even though the idea is quite old (compilers are a form of software generator),
   this idea has not been picked up in mainstream software development.
   Yet, research so far has yielded promising results with respect to validation and resilience.
   A software generator enables the DevOps way of working, which is to demonstrate the end product at the end of every sprint.
   Thus, a way of software development emerges that allows for very fast incremental development of large information systems.
   Research at the Open Universiteit has produced software generators specifically for information systems.
   This is useful for supporting distributed business processes that require a mix of human and automated activity.

\section{Benefits}
    The Open University offers its expertise~\cite{Joosten-JLAMP2018,JoostenRAMiCS2017,10.1145/3354166.3354182} and tools as a building block for a larger research effort,
    in which generative software development is an enabler for the research proposed.
    This capability offers the following benefits:
\begin{itemize}
    \item By generating prototypes, we can produce live demonstrations periodically to keep the future user population informed of the progress.
    This produces quick results, which keeps funders happy.
    \item By automating software production, we can prevent a large class of mistakes and thereby reduce the project risk.
    \item By formally specifying the functionality, we can give valuable guarantees on the correctness of the results.
    \item Software that is correctly generated needs no testing. This yields significant savings on testing and test maintenance.
\end{itemize}

\section{Researchers}
    The authors are acknowledged scientists, both with a proven track record in generative software development.

    \section{Bibliography}
\bibliographystyle{elsarticle-harv}
\bibliography{doc}
    

\end{document}