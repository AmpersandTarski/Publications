\documentclass[11pt,a4paper]{article}
\usepackage{isabelle,isabellesym}
\usepackage{amsfonts, amsmath, amssymb}

% this should be the last package used
\usepackage{pdfsetup}

% urls in roman style, theory text in math-similar italics
\urlstyle{rm}
\isabellestyle{it}


\begin{document}
\title{Proofs for Preserving Invariant Rules in Relation Algebra}
\author{Stef Joosten (0000-0001-8308-0189)}
%orcid.org/0000-0001-8308-0189
\maketitle

\begin{abstract}
	Relation Algebra can be used as a programming language for building information systems.
	This claim was made in a paper~\cite{JoostenRAMiCS2017} at the conference
	``Relational and Algebraic Methods in Computer Science''.
	An entirely generated information system was demonstrated and presented in Lyon on May 17th of 2017 at the 16th edition of the conference.
	It has been implemented in a tool called Ampersand,
	in which programming is done by specifying invariant rules.
	An invariant is a rule that must remain satisfied as long as it is valid in its context.
	This paper contains the proofs that are needed to build the code for such software.
	It is a companion paper to the extended version of the RAMiCS-2017 paper~\cite{Joosten-AFP17}.
\end{abstract}

\tableofcontents

\section{Introduction}
\label{sct:Introduction}
	This paper presents proofs that are used for programming information systems in relation algebra.
	A compiler, Ampersand~\cite{Michels2011}, is used to compile concepts, relations and rules into a working database-application.
	Ampersand is a syntactically sugared version of heterogeneous relation algebra~\cite{Schmidt1997}.
	
	The user may regard Ampersand as a programming language, especially suited for designing the back-end of information systems.
	The axioms of Tarski can be used to manipulate expressions in a way that preserves meaning~\cite{vdWoude2011}.
	This makes Ampersand a declarative language.

	This paper contributes by publishing the required proofs in the Archive of Formal Proofs\footnote{https://www.isa-afp.org/}.
	It means that every proof presented here has been validated in the theorem prover Isabelle~\cite{Nipkow2002}.
	These proofs contribute to a theory to assist in preserving invariants.

\input{session}

\bibliographystyle{abbrv}
\bibliography{root}

\end{document}

%%% Local Variables:
%%% mode: latex
%%% TeX-master: t
%%% End:
